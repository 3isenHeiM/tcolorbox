% !TeX root = tcolorbox.tex
% include file of tcolorbox.tex (manual of the LaTeX package tcolorbox)
\clearpage
\section{Libraries
  \mylib{listings},
  \mylib{listingsutf8}, and
  \mylib{minted}}\label{sec:listings}%
\tcbset{external/prefix=external/listings_}%

\subsection{Loading the Libraries}
In contrast to other |tcolorbox| libraries, the libraries
\mylib{listings}, \mylib{listingsutf8}, and \mylib{minted} are concurrent in the sense that
they all do the same thing, i.\,e.\ displaying listings with or without typesetting
the listing in \LaTeX\ parallel.
The difference is the underlying \LaTeX\ package which does the core job for
displaying a listing. So, typically, you need just \emph{one} of these
libraries. If you do not have a clue which one of them you should use
and you are using |pdflatex|, you should take \mylib{listingsutf8}.
If you are using |xelatex| or |lualatex|, you should take \mylib{listings}
as |xelatex| and |lualatex| are not compatible with \mylib{listingsutf8}.

\begin{marker}
The order in which the libraries are included influences the default settings and
the \refKey{/tcb/reset} behavior. The settings of a later loaded library overwrite
the settings of a previous loaded library. A library is never loaded twice.
\end{marker}


\subsubsection{Loading \mylib{listings}}
This library uses the package |listings| \cite{hoffmann:listings} to typeset
listings. It is loaded by a package option or inside the preamble by:
\begin{dispListing}
\tcbuselibrary{listings}
\end{dispListing}
This also loads the package |listings| \cite{hoffmann:listings}.

The \refKey{/tcb/listing engine} is set to |listings| by the library.
To reactivate this setting, if overwritten by other libraries, use
\begin{dispListing}
\tcbset{listing engine=listings}
\end{dispListing}

\subsubsection{Loading \mylib{listingsutf8}}
\begin{marker}
This library is not needed (and troublesome) when using Xe\LaTeX\ or Lua\LaTeX.
Therefore, loading this library is automatically replaced by loading
\mylib{listings} only, if pdf\LaTeX\ is \emph{not} used.
\end{marker}
To extend |listings| for UTF-8 encoded sources, you can use the support from
the package |listingsutf8| \cite{oberdiek:listingsutf8} by loading the library
variant \mylib{listingsutf8}.
\begin{dispListing}
\tcbuselibrary{listingsutf8}
\tcbset{listing utf8=latin1}% optional; 'latin1' is the default.
\end{dispListing}
This also loads the library \mylib{listings}
and the packages |listings| \cite{hoffmann:listings}
and |listingsutf8| \cite{oberdiek:listingsutf8}.

The \refKey{/tcb/listing engine} is set to |listings| by the library.
To reactivate this setting, if overwritten by other libraries, use
\begin{dispListing}
\tcbset{listing engine=listings}
\end{dispListing}


\clearpage
\subsubsection{Loading \mylib{minted}}
This library uses the package |minted| \cite{poore:minted} to typeset
listings. It is loaded by a package option or inside the preamble by:
\begin{dispListing}
\tcbuselibrary{minted}
\end{dispListing}
This also loads the package |minted| \cite{poore:minted}.

\begin{marker}
The |minted| package uses the external tool |Pygments| \cite{pygments:web}
to apply syntax highlighting. It has to be installed and set up, before the
library can be used, see \cite{poore:minted} and \cite{pygments:web}.
The |tcolorbox| library \mylib{minted} does not work, if the package
|minted| \cite{poore:minted} does not work.
\end{marker}

The \refKey{/tcb/listing engine} is set to |minted| by the library.
To reactivate this setting, if overwritten by other libraries, use
\begin{dispListing}
\tcbset{listing engine=minted}
\end{dispListing}


\subsection{Common Macros of the Libraries}

\begin{docEnvironment}{tcblisting}{\marg{options}}
  Creates a colored box based on a \refEnv{tcolorbox}.
  Controlled by the given \meta{options}, the
  environment content is typeset normally and/or as a listing.
  Furthermore, the \meta{options} control appearance and functions of
  the |tcolorbox|.
  By default, the listing is interpreted as a \LaTeX\ listing.
\begin{dispExample}
\begin{tcblisting}{colback=red!5!white,colframe=red!75!black}
This is a \LaTeX\ example which displays the text as source code
and in compiled form.
\end{tcblisting}
\end{dispExample}

\clearpage

\begin{dispExample}
% \tcbuselibrary{listings} /or/ \tcbuselibrary{listingsutf8}
\begin{tcblisting}{colback=yellow!5,colframe=yellow!50!black,listing only,
    title=This is source code in another language (XML), fonttitle=\bfseries,
  listing options={language=XML,columns=fullflexible,keywordstyle=\color{red}}}
<?xml version="1.0"?>
<project name="Package tcolorbox" default="documentation" basedir=".">
  <description>
    Apache Ant build file (http://ant.apache.org/)
  </description>
</project>
\end{tcblisting}
\end{dispExample}

\enlargethispage*{10mm}

\begin{dispExample}
% \tcbuselibrary{minted}
\begin{tcblisting}{colback=yellow!5,colframe=yellow!50!black,listing only,
    title=This is source code in another language (XML), fonttitle=\bfseries,
  listing engine=minted,minted language=xml}
<?xml version="1.0"?>
<project name="Package tcolorbox" default="documentation" basedir=".">
  <description>
    Apache Ant build file (http://ant.apache.org/)
  </description>
</project>
\end{tcblisting}
\end{dispExample}



\begin{dispExample}
% This box is as wide as needed (listing only !!)
% \tcbuselibrary{skins}
\begin{tcblisting}{colback=green!5!white,colframe=green!50!black,listing only,
  hbox,enhanced,drop fuzzy shadow,before=\begin{center},after=\end{center}}
\begin{tikzpicture}
\fill[red] (0,0) rectangle (1,1);
\end{tikzpicture}
\end{tcblisting}
\end{dispExample}
\end{docEnvironment}


\clearpage
\begin{docEnvironment}{tcboutputlisting}{}
  Saves the environment content to a file which is named by the key value of
  |listing file|. Later, this file can be loaded by
  |\tcbinputlisting| or |\tcbuselistingtext| or |\tcbuselistinglisting|.
\begin{dispListing}
\begin{tcboutputlisting}
This \textbf{text} is written to a standardized file for later usage.
\end{tcboutputlisting}
\end{dispListing}
\end{docEnvironment}


\begin{docCommand}{tcbinputlisting}{\marg{options}}
  Creates a colored boxed based on a |tcolorbox|. The text content is read
  from a file named by the key value of |listing file|. Apart from that,
  the function is equal to that of \refEnv{tcblisting}.
\begin{dispExample}
\tcbinputlisting{colback=red!5!white,colframe=red!75!black,text only}
\tcbinputlisting{colback=green!5,colframe=green!75!black,listing only}
\end{dispExample}
\end{docCommand}

\begin{docCommand}{tcbuselistingtext}{}
  Loads text from a file named by the key value of |listing file|.
\begin{dispExample}
\tcbuselistingtext
\end{dispExample}
\end{docCommand}


\begin{docCommand}{tcbuselistinglisting}{}
  Typesets text as listing from a file named by the key value of |listing file|.
\begin{dispExample}
\tcbuselistinglisting
\end{dispExample}
\end{docCommand}

%\enlargethispage*{5mm}
\begin{docCommand}{tcbusetemplisting}{}
  Typesets text as listing from a temporary file which was written by
  \refEnv{tcbwritetemp}.
\end{docCommand}


\clearpage
\begin{marker}
See \Vref{subsec:xparse_listing} and \Vref{subsec:xparse_inputlisting} for more
elaborate methods to create new environments and commands.
\end{marker}
\begin{marker}
If a new sort of |tcblisting| environments should be created with
one optional argument only, one is highly recommended to use
\refCom{DeclareTCBListing} or \refCom{NewTCBListing}
instead of \refCom{newtcblisting} to
avoid content scanning problems.
\end{marker}

\begin{docCommand}{newtcblisting}{\oarg{init options}\marg{name}\oarg{number}\oarg{default}\marg{options}}
  Creates a new environment \meta{name} based on \refEnv{tcblisting}.
  Basically, |\newtcblisting| operates like |\newenvironment|. This means,
  the new environment \meta{name} optionally takes \meta{number} arguments, where
  \meta{default} is the default value for the optional first argument.
  The \meta{options} are given to the underlying |tcblisting|.
  Note that \refKey{/tcb/savedelimiter} is set to the given \meta{name}
  automatically.
  The \meta{init options} allow setting up automatic numbering,
  see Section \ref{sec:initkeys} from page \pageref{sec:initkeys}.
\begin{dispExample*}{sbs,lefthand ratio=0.5}
\newtcblisting{mybox}{%
  colback=red!5!white,
  colframe=red!75!black}

\begin{mybox}
This is my \LaTeX\ box.
\end{mybox}
\end{dispExample*}

\begin{dispExample*}{sbs,lefthand ratio=0.5}
\newtcblisting{mybox}[1]{%
  colback=red!5!white,
  colframe=red!75!black,
  fonttitle=\bfseries,
  title=#1}

\begin{mybox}{Listing Box}
This is my \LaTeX\ box.
\end{mybox}
\end{dispExample*}

\begin{dispExample*}{sbs,lefthand ratio=0.5}
\newtcblisting{mybox}[2][]{%
  colback=red!5!white,
  colframe=red!75!black,
  fonttitle=\bfseries,
  title=#2,#1}

\begin{mybox}[listing only]
  {Listing Box}
This is my \LaTeX\ box.
\end{mybox}
\bigskip

\begin{mybox}[listing side text]
  {Listing Box}
This is my
\LaTeX\ box.
\end{mybox}
\end{dispExample*}

\clearpage
\inputpreamblelisting{C}

\begin{dispExample*}{sbs,lefthand ratio=0.5}
\begin{mycbox}{Listing Box}
This is my \LaTeX\ box.
\end{mycbox}
\end{dispExample*}
\end{docCommand}


%\enlargethispage*{1cm}
\begin{docCommand}{renewtcblisting}{\oarg{init options}\marg{name}\oarg{number}\oarg{default}\marg{options}}
  Operates like \refCom{newtcblisting}, but based on |\renewenvironment| instead of |\newenvironment|.
  An existing environment is redefined.
\end{docCommand}


\clearpage
\begin{docCommand}{newtcbinputlisting}{\oarg{init options}\brackets{\texttt{\textbackslash}\rmfamily\meta{name}}\oarg{number}\oarg{default}\marg{options}}
  Creates a new macro \texttt{\textbackslash}\meta{name} based on \refCom{tcbinputlisting}.
  Basically, |\newtcbinputlisting| operates like |\newcommand|.
  The new macro \texttt{\textbackslash}\meta{name} optionally takes \meta{number} arguments, where
  \meta{default} is the default value for the optional first argument.
  The \meta{options} are given to the underlying |tcbinputlisting|.
  The \meta{init options} allow setting up automatic numbering,
  see Section \ref{sec:initkeys} from page \pageref{sec:initkeys}.

\begin{dispExample}
\newtcbinputlisting[use counter from=mycbox]{\mylisting}[2][]{%
  listing file={#2},
  title=Listing (\thetcbcounter) of \texttt{#2},
  colback=red!5!white,colframe=red!75!black,fonttitle=\bfseries,
  listing only,breakable,#1}

\mylisting[before upper=\textit{This is the included file content:}]
          {\jobname.tcbtemp}
\end{dispExample}

\begin{dispExample}
\newtcbinputlisting[use counter from=mycbox]{\mylisting}[2][]{%
  listing engine=minted,minted language=latex,minted style=colorful,
  listing file={#2},
  title=Listing (\thetcbcounter) of \texttt{#2},
  colback=red!5!white,colframe=red!75!black,fonttitle=\bfseries,
  listing only,breakable,#1}

\mylisting[before upper=\textit{This is the included file content:}]
          {\jobname.tcbtemp}
\end{dispExample}
\end{docCommand}


\begin{docCommand}{renewtcbinputlisting}{\oarg{init options}\brackets{\texttt{\textbackslash}\rmfamily\meta{name}}\oarg{number}\oarg{default}\marg{options}}
  Operates like \refCom{newtcbinputlisting}, but based on |\renewcommand| instead of |\newcommand|.
  An existing macro is redefined.
\end{docCommand}


\clearpage
\subsection{Option Keys of the \mylib{listings} Library}\label{sec:speclistingkeys}

\begin{docTcbKey}{listing options}{=\meta{key list}}{no default, initially |style=tcblatex|}
  Sets the options from the package |listings| \cite{hoffmann:listings}
  which are used during typesetting of the listing.
  For \LaTeX\ listings, there is a predefined |listings| style named |tcblatex|
  which can be used.
\begin{dispExample}
\begin{tcblisting}{colback=red!5!white,colframe=red!25,left=6mm,
listing options={style=tcblatex,numbers=left,numberstyle=\tiny\color{red!75!black}}}
This is a \LaTeX\ example which displays the text as source code
and in compiled form. Additionally, we use line numbers here.
\end{tcblisting}
\end{dispExample}
\end{docTcbKey}


\begin{docTcbKey}{no listing options}{}{no value, initially unset}
  Abbreviation for |listing options={}|.
  This removes all options for the |listings| package.
  This includes the |tcblisting| standard style |tcblatex| and the encoding presets.
  Use this option, if you want to set the |listings| options outside of |tcblisting|, e.\,g.\ globally in
  the preamble.
\begin{dispExample}
\begin{tcblisting}{no listing options}
All \textit{listings} options removed.
\end{tcblisting}
\end{dispExample}
\end{docTcbKey}


\begin{docTcbKey}{listing style}{=\meta{style}}{no default, initially |tcblatex|}
  Abbreviation for |listing options={style=...}|. This key sets a \meta{style}
  for the |listings| package, see \cite{hoffmann:listings}.
  For \LaTeX, there is a predefined style named |tcblatex|.
\begin{dispExample}
\begin{tcblisting}{colback=red!5!white,colframe=red!75!black,
listing style=tcblatex}
Here, we use the predefined style.
\end{tcblisting}
\end{dispExample}
\end{docTcbKey}

\clearpage
\begin{docTcbKey}{listing inputencoding}{=\meta{encoding}}{no default, initially \texttt{\cs inputencodingname}}
  Sets the input encoding value for the predefined listing style |tcblatex|
  and |tcbdocumentation| from the library \mylib{documentation}.
  The initial value is derived from the package |inputenc| if used.
\end{docTcbKey}

\begin{docTcbKey}{listing remove caption}{\colOpt{=true\textbar false}}{default |true|, initially |true|}
If set to |true|, some part of the caption building code of the |listings| package
is silenced to prevent some unwanted interaction with the |hyperref| package resulting
in additional vertical space.
If set to |false|, the |listings| package code is kept unchanged.
Note that listings outside \refEnv{tcblisting} and
\refCom{tcbinputlisting} are always processed normally.
Typically, a user is not expected to use this key at all.
\end{docTcbKey}

\begin{docTcbKey}{every listing line}{=\meta{text}}{no default, initially unset/empty}
Inserts some \meta{text} to the begin of every line of a listing.
Note that this a hack of the |listings| package code. This may become unusable
or superfluous in the future.
\begin{dispExample}
\newtcblisting{commandshell}{colback=black,colupper=white,colframe=yellow!75!black,
  listing only,listing options={style=tcblatex,language=sh},
  every listing line={\textcolor{red}{\small\ttfamily\bfseries root \$> }}}

\begin{commandshell}
ls -al
cd /usr/lib
\end{commandshell}
\end{dispExample}
\end{docTcbKey}


\begin{docTcbKey}{every listing line*}{=\meta{text}}{no default, initially unset/empty}
Identical to \refKey{/tcb/every listing line} plus additional enlargement
of \refKey{/tcb/rightupper} by the width of \meta{text}. Therefore, this
option has to be used after the geometry settings are done.
This option is intended to be used in conjunction with \refKey{/tcb/hbox}.

\begin{dispExample}
\newtcblisting{commandshell}{colback=black,colupper=white,colframe=yellow!75!black,
  listing only,listing options={style=tcblatex,language=sh},hbox,
  every listing line*={\textcolor{red}{\small\ttfamily\bfseries root \$> }}}

\begin{commandshell}
ls -al
cd /usr/lib
\end{commandshell}
\end{dispExample}
\end{docTcbKey}

See further options in \Vref{sec:commonlistingkeys}.

\begin{marker}
For an combined example of using |\lstinline| inside a |tcolorbox|, see
\refCom{DeclareTotalTCBox}.
\end{marker}

\clearpage
\subsection{Option Keys of the \mylib{listingsutf8} Library}
\begin{marker}
The \mylib{listingsutf8} library is not needed (and troublesome) when using Xe\LaTeX\ or Lua\LaTeX.
Therefore, loading this library is automatically replaced by loading
\mylib{listings} only, if pdf\LaTeX\ is \emph{not} used.
\end{marker}

The \mylib{listingsutf8} library is an extension of the
\mylib{listings} library, so
all options from \Vref{sec:speclistingkeys} are applicable.

\begin{docTcbKey}{listing utf8}{=\meta{one-byte-encoding}}{style, no default, initially |latin1|}
  Abbreviation for using \refKey{/tcb/listing inputencoding}
  together with UTF-8 support from the package |listingsutf8| \cite{oberdiek:listingsutf8}.
  This option is available only for the library variant \mylib{listingsutf8}.
  The \meta{one-byte-encoding} is one of
  the applicable encodings from \cite{oberdiek:listingsutf8}, e.\,g.\ |latin1|
  which is the default.\par
  Be aware that this means restriction to this specific \meta{one-byte-encoding}:
  e.\,g.\ |latin1| comprises umlauts and other accented characters, but not
  the Euro sign. If you want to use the |listings| package \emph{and} \flqq real\frqq\
  UTF-8 source code, then do \emph{not} use \mylib{listingsutf8} but \mylib{listings}
  with
  \refKey{/tcb/listing inputencoding}|=utf8|
  \emph{and} with specific manual hacks for specific UTF-8-encoded characters.

\end{docTcbKey}

See further options in \Vref{sec:commonlistingkeys}.

\clearpage
\subsection{Option Keys of the \mylib{minted} Library}

\begin{docTcbKey}{minted language}{=\meta{programming language}}{no default, initially |latex|}
Sets a \meta{programming language} known to |Pygments| \cite{pygments:web}.
\begin{dispExample}
\begin{tcblisting}{listing engine=minted,minted style=trac,
  minted language=java,
  colback=red!5!white,colframe=red!75!black,listing only}
    public class HelloWorld {
      // A 'Hello World' in Java
      public static void main(String[] args) {
        System.out.println("Hello World!");
      }
    }
\end{tcblisting}
\end{dispExample}
\end{docTcbKey}


\begin{docTcbKey}[][doc updated={2017-07-05}]{minted options}{=\meta{key list}}{no default, initially
    |tabsize=2,fontsize=\textbackslash small,|\linebreak|breaklines,autogobble|}
  Sets the options from the package |minted| \cite{poore:minted}
  which are used during typesetting of the listing.
\begin{dispExample}
% \tcbuselibrary{skins}
\newtcblisting{myjava}{listing engine=minted,
  minted style=colorful,
  minted language=java,
  minted options={fontsize=\small,breaklines,autogobble,linenos,numbersep=3mm},
  colback=blue!5!white,colframe=blue!75!black,listing only,
  left=5mm,enhanced,
  overlay={\begin{tcbclipinterior}\fill[red!20!blue!20!white] (frame.south west)
    rectangle ([xshift=5mm]frame.north west);\end{tcbclipinterior}}}

\begin{myjava}
  public class HelloWorld {
    // A 'Hello World' in Java
    public static void main(String[] args) {
      System.out.println("Hello World!");
    }
  }
\end{myjava}
\end{dispExample}
\end{docTcbKey}

\clearpage
\begin{docTcbKey}{minted style}{=\meta{style}}{no default, initially unset}
Sets a \meta{style} known to |Pygments| \cite{pygments:web}. This is
independent from \refKey{/tcb/minted options}. Note that styles are always
applied globally; all following examples will be set in the given \meta{style}
until a new style is set. Also note that
setting |\usemintedstyle|\marg{style} only once per document is more economic, if
all styles in a document are the same.
For examples of different styles, see
\refKey{/tcb/minted language} and \refKey{/tcb/minted options}.
\end{docTcbKey}

See further options in \Vref{sec:commonlistingkeys}.


\clearpage
\subsection{Common Option Keys of all Libraries}\label{sec:commonlistingkeys}
For the \meta{options} in \refEnv{tcblisting} respectively \refCom{tcbinputlisting}
the following |pgf| keys can be applied. The key tree path |/tcb/| is not to
be used inside these macros.

\begin{docTcbKey}{listing engine}{=\meta{engine}}{no default}
  Sets the \meta{engine} which typesets the listings. Feasible values are
  \begin{itemize}
  \item\docValue{listings}, if library \mylib{listings} or
  \mylib{listingsutf8} is loaded.
  \item\docValue{minted}, if library \mylib{minted} is loaded.
  \end{itemize}
\end{docTcbKey}

\begin{docTcbKey}{listing file}{=\meta{file name}}{no default, initially \cs{jobname.listing}}
  Sets the \meta{file name} of the file which is used to save listings.
\end{docTcbKey}


\begin{docTcbKey}{listing and text}{}{no value, initially set}
  Typesets the environment content as listing in the upper part and
  as compiled text in the lower part.
\begin{dispExample}
\begin{tcblisting}{colback=red!5!white,colframe=red!75!black,listing and text}
This is a \LaTeX\ example.
\end{tcblisting}
\end{dispExample}
\end{docTcbKey}


\begin{docTcbKey}{text and listing}{}{no value}
  Typesets the environment content as compiled text in the upper part and
  as listing in the lower part.
\begin{dispExample}
\begin{tcblisting}{colback=red!5!white,colframe=red!75!black,text and listing}
This is a \LaTeX\ example.
\end{tcblisting}
\end{dispExample}
\end{docTcbKey}


\begin{docTcbKey}{listing only}{}{no value}
  Typesets the environment content as listing.
\begin{dispExample}
\begin{tcblisting}{colback=red!5!white,colframe=red!75!black,listing only}
This is a \LaTeX\ example.
\end{tcblisting}
\end{dispExample}
\end{docTcbKey}


\clearpage
\begin{docTcbKey}{text only}{}{no value}
  Typesets the environment content as compiled text.
\begin{dispExample}
\begin{tcblisting}{colback=red!5!white,colframe=red!75!black,text only}
This is a \LaTeX\ example.
\end{tcblisting}
\end{dispExample}
\end{docTcbKey}



\begin{docTcbKey}{comment}{=\meta{text}}{no default, initially empty}
  Records a comment with \meta{text} as content. The comment is displayed
   e.\,g.\ in conjunction with \refKey{/tcb/listing and comment}
  and \refKey{/tcb/comment and listing}.
\begin{dispExample}
\begin{tcblisting}{comment={This comment is really only a comment},
  colback=red!5!white,colframe=red!75!black}
This is a \textbf{tcolorbox}.
\end{tcblisting}
\end{dispExample}
\end{docTcbKey}


\begin{docTcbKey}[][doc new=2014-11-17]{comment only}{}{no value}
  Typesets the environment content with the comment text.
\begin{dispExample}
\begin{tcblisting}{comment only,
  comment={This is a comment.},
  colback=red!5!white,colframe=red!75!black}
This is a \textbf{tcolorbox}.
\end{tcblisting}
\end{dispExample}
\end{docTcbKey}



\begin{docTcbKey}{image comment}{=\marg{options}\marg{filename}}{style, no default, initially unset}
  Uses an image denoted by \meta{filename} as \textit{comment} for the listing.
  The image is included by the standard |\includegraphics| macro with
  given \meta{options}.
\begin{dispExample}
\begin{tcblisting}{colback=red!5!white,colframe=red!75!black,listing side comment,
  image comment={width=2.5cm}{example-image-a.pdf},center lower}
This is a \LaTeX\ example.
\end{tcblisting}
\end{dispExample}
\end{docTcbKey}


\clearpage
\begin{docTcbKey}[][doc new=2014-11-14]{tcbimage comment}{=\meta{filename}}{style, no default, initially unset}
  Uses an image denoted by \meta{filename} as \textit{comment} for the listing.
  The image is included by the \refCom{tcbincludegraphics} macro.
  The inclusion can be customized by \refKey{/tcb/comment style}.
  \begin{marker}
  The library \mylib{skins} is needed to apply this option.
  \end{marker}
  \medskip
\begin{dispExample}
% \tcbuselibrary{skins}
\begin{tcblisting}{colback=red!5!white,colframe=red!75!black,listing side comment,
  righthand width=3cm,lower separated=false,
  tcbimage comment={example-image-a.pdf},
  comment style={size=fbox,colframe=blue,colback=blue!50,sharp corners,
    drop fuzzy shadow}}
This is a \LaTeX\ example.
\end{tcblisting}
\end{dispExample}
\end{docTcbKey}

\clearpage

\begin{docTcbKey}[][doc new=2014-11-14]{pdf comment}{\colOpt{=\meta{filename}}}{style, default listing file, initially unset}
  Uses a PDF file denoted by \meta{filename} as \textit{comment} for the listing.
  The image is included by \refCom{tcbincludepdf} inside a \refEnv{tcbraster}.
  The inclusion can be customized by \refKey{/tcb/comment style}.
  \begin{marker}
  The libraries \mylib{skins} and \mylib{raster} are needed to apply this option.
  \end{marker}
  \medskip
\begin{dispExample}
% \tcbuselibrary{skins,raster}
\begin{tcblisting}{colback=red!5!white,colframe=red!75!black,listing and comment,
  righthand width=3cm,lower separated=false,middle=1mm,
  pdf comment={tcolorbox-example.pdf},
  comment style={raster columns=3,graphics pages={1,2,3},
    colframe=blue,drop fuzzy shadow}}
This is a \LaTeX\ example.
\end{tcblisting}
\end{dispExample}
\end{docTcbKey}

\clearpage


\begin{docTcbKey}[][doc new=2014-11-14]{pdf extension}{=\meta{extension}}{no default, initially |pdf|}
  Sets the PDF file name extension for \refKey{/tcb/pdf comment} to \meta{extension}.
  Note that \meta{extension} always overwrites any actual extension given
  inside \refKey{/tcb/pdf comment}.
\end{docTcbKey}


\begin{docTcbKey}[][doc new=2014-11-14]{comment style}{=\meta{options}}{no default, initially empty}
  Sets the \meta{options} for \refKey{/tcb/tcbimage comment} and \refKey{/tcb/pdf comment}.
  These are |tcolorbox| options to customize the colored box drawn around the
  image(s), also image options encapsulated by \refKey{/tcb/graphics options},
  and \refEnv{tcbraster} options for \refKey{/tcb/pdf comment}.
\end{docTcbKey}


\begin{docTcbKey}{listing and comment}{}{no value}
  Typesets the environment content as listing in the upper part and
  a given comment in the lower part.
\begin{dispExample}
\begin{tcblisting}{colback=red!5!white,colframe=red!75!black,listing and comment,
  comment={This is my comment. It may contain line breaks.\par
    It can even use the environment content
    \flqq\ignorespaces\tcbuselistingtext\unskip\frqq}}
This is a \LaTeX\ example.
\end{tcblisting}
\end{dispExample}
\end{docTcbKey}

\enlargethispage*{10mm}
\begin{docTcbKey}{comment and listing}{}{no value}
  Typesets a given comment in the upper part and
  the environment content as listing in the lower part.
\begin{dispExample}
\begin{tcblisting}{colback=red!5!white,colframe=red!75!black,comment and listing,
comment={This is my comment.}}
This is a \LaTeX\ example.
\end{tcblisting}
\end{dispExample}
\end{docTcbKey}


\clearpage
\begin{docTcbKey}{listing side text}{}{style, no value}
Typesets the environment content side by side as listing in the left (upper)
part and as compiled text in the right (lower) part.
This is a shortcut for setting \refKey{/tcb/listing and text} and \refKey{/tcb/sidebyside}.
\begin{dispExample}
\begin{tcblisting}{colback=red!5!white,colframe=red!75!black,listing side text}
This is a \LaTeX\ example.
\end{tcblisting}
\end{dispExample}
\begin{marker}
Note that |sidebyside=false| has to be added, if the setting of
\refKey{/tcb/listing side text} is to be annihilated.
\end{marker}
\end{docTcbKey}


\begin{docTcbKey}{text side listing}{}{style, no value}
Typesets the environment content side by side as compiled text in the left (upper)
part and as listing in the right (lower) part.
This is a shortcut for setting \refKey{/tcb/text and listing} and \refKey{/tcb/sidebyside}.
\begin{dispExample}
\begin{tcblisting}{colback=red!5!white,colframe=red!75!black,text side listing}
This is a \LaTeX\ example.
\end{tcblisting}
\end{dispExample}
\end{docTcbKey}


\begin{docTcbKey}{listing outside text}{}{no value}
Typesets the environment content side by side as listing in a |tcolorbox|
and as compiled text outside the box in the right part of the page.
Nevertheless, the outside text is treated as \emph{lower} part of the
|tcolorbox| and can be formatted with all lower part options.
The space partitioning is done with the side by side options from
\Fullref{sec:sidebyside}.
\begin{dispExample}
\begin{tcblisting}{colback=red!5!white,colframe=red!75!black,listing outside text}
This is a \LaTeX\ example.
\end{tcblisting}
\end{dispExample}
\end{docTcbKey}

\clearpage

\begin{docTcbKey}{text outside listing}{}{no value}
Typesets the environment content side by side as listing in a |tcolorbox|
and as compiled text outside the box in the left part of the page.
Nevertheless, the outside text is treated as \emph{lower} part of the
|tcolorbox| and can be formatted with all lower part options.
The space partitioning is done with the side by side options from
\Fullref{sec:sidebyside}.
\begin{dispExample}
\begin{tcblisting}{colback=red!5!white,colframe=red!75!black,text outside listing}
This is a \LaTeX\ example.
\end{tcblisting}
\end{dispExample}
\end{docTcbKey}



\begin{docTcbKey}{listing side comment}{}{style, no value}
Typesets the environment content side by side as listing in the left (upper)
part and a given comment in the right (lower) part.
This is a shortcut for setting \refKey{/tcb/listing and comment} and \refKey{/tcb/sidebyside}.
\begin{dispExample}
\begin{tcblisting}{colback=red!5!white,colframe=red!75!black,listing side comment,
  righthand width=1.5cm,image comment={width=1.5cm}{example-image-a.pdf}}
This is a \LaTeX\ example.
\end{tcblisting}
\end{dispExample}
\end{docTcbKey}


\begin{docTcbKey}{comment side listing}{}{style, no value}
Typesets the environment content side by side with a given comment in the left (upper)
part and as listing in the right (lower) part.
This is a shortcut for setting \refKey{/tcb/comment and listing} and \refKey{/tcb/sidebyside}.
\begin{dispExample}
\begin{tcblisting}{colback=red!5!white,colframe=red!75!black,comment side listing,
  lefthand width=1.5cm,image comment={width=1.5cm}{example-image-a.pdf}}
This is a \LaTeX\ example.
\end{tcblisting}
\end{dispExample}
\end{docTcbKey}

\clearpage

\begin{docTcbKey}{listing outside comment}{}{no value}
Typesets the environment content side by side as listing in a |tcolorbox|
and a given comment outside the box in the right part of the page.
Nevertheless, the outside text is treated as \emph{lower} part of the
|tcolorbox| and can be formatted with all lower part options.
The space partitioning is done with the side by side options from
\Fullref{sec:sidebyside}.
\begin{dispExample}
\begin{tcblisting}{colback=red!5!white,colframe=red!75!black,listing outside comment,
  righthand width=1.5cm,image comment={width=1.5cm}{example-image-a.pdf}}
This is a \LaTeX\ example.
\end{tcblisting}
\end{dispExample}
\end{docTcbKey}


\begin{docTcbKey}{comment outside listing}{}{no value}
Typesets the environment content side by side as listing in a |tcolorbox|
and a given comment outside the box in the left part of the page.
Nevertheless, the outside text is treated as \emph{lower} part of the
|tcolorbox| and can be formatted with all lower part options.
The space partitioning is done with the side by side options from
\Fullref{sec:sidebyside}.
\begin{dispExample}
\begin{tcblisting}{colback=red!5!white,colframe=red!75!black,comment outside listing,
  lefthand width=1.5cm,image comment={width=1.5cm}{example-image-a.pdf}}
This is a \LaTeX\ example.
\end{tcblisting}
\end{dispExample}
\end{docTcbKey}



\begin{docTcbKey}{listing above text}{}{no value}
  Typesets the environment content as listing in a |tcolorbox|
  and as compiled text outside and below the box.
  The outside text is treated as \emph{lower} part of the
  |tcolorbox| and can be formatted with all lower part options.
  The distance between box and text is controlled by \refKey{/tcb/middle}.
\begin{dispExample}
\begin{tcblisting}{colback=red!5!white,colframe=red!75!black,listing above text}
This is a \LaTeX\ example.
\end{tcblisting}
\end{dispExample}
\end{docTcbKey}


\begin{docTcbKey}[][doc new=2014-11-07]{listing above* text}{}{no value}
  Widely equal to \refKey{/tcb/listing above text}, but the outside text is
  not formatted with the lower part options.
  Also, it is not put into a minipage and it may span several pages.
  The distance between box and text is controlled by \refKey{/tcb/after}.
\end{docTcbKey}

\clearpage

\begin{docTcbKey}{text above listing}{}{no value}
  Typesets the environment content as listing in a |tcolorbox|
  and as compiled text outside and above the box.
  The outside text is treated as \emph{lower} part of the
  |tcolorbox| and can be formatted with all lower part options.
  The distance between box and text is controlled by \refKey{/tcb/middle}.
\begin{dispExample}
\begin{tcblisting}{colback=red!5!white,colframe=red!75!black,text above listing}
This is a \LaTeX\ example.
\end{tcblisting}
\end{dispExample}
\end{docTcbKey}


\begin{docTcbKey}[][doc new=2014-11-07]{text above* listing}{}{no value}
  Widely equal to \refKey{/tcb/text above listing}, but the outside text is
  not formatted with the lower part options.
  Also, it is not put into a minipage and it may span several pages.
  The distance between box and text is controlled by \refKey{/tcb/before}.
\end{docTcbKey}



\begin{docTcbKey}{listing above comment}{}{no value}
  Typesets the environment content as listing in a |tcolorbox|
  and a given comment outside and below the box.
  The outside text is treated as \emph{lower} part of the
  |tcolorbox| and can be formatted with all lower part options.
  The distance between box and comment is controlled by \refKey{/tcb/middle}.
\begin{dispExample}
\begin{tcblisting}{colback=red!5!white,colframe=red!75!black,listing above comment,
  center lower,image comment={width=3cm}{example-image-a.pdf}}
This is a \LaTeX\ example.
\end{tcblisting}
\end{dispExample}
\end{docTcbKey}


\begin{docTcbKey}[][doc new=2014-11-07]{listing above* comment}{}{no value}
  Widely equal to \refKey{/tcb/listing above comment}, but the outside comment is
  not formatted with the lower part options.
  Also, it is not put into a minipage and it may span several pages.
  The distance between box and comment is controlled by \refKey{/tcb/after}.
\end{docTcbKey}

\clearpage

\begin{docTcbKey}{comment above listing}{}{no value}
  Typesets the environment content as listing in a |tcolorbox|
  and a given comment outside and above the box.
  The outside text is treated as \emph{lower} part of the
  |tcolorbox| and can be formatted with all lower part options.
  The distance between box and comment is controlled by \refKey{/tcb/middle}.
\begin{dispExample}
\begin{tcblisting}{colback=red!5!white,colframe=red!75!black,comment above listing,
  center lower,image comment={width=3cm}{example-image-a.pdf}}
This is a \LaTeX\ example.
\end{tcblisting}
\end{dispExample}
\end{docTcbKey}


\begin{docTcbKey}[][doc new=2014-11-07]{comment above* listing}{}{no value}
  Widely equal to \refKey{/tcb/comment above listing}, but the outside comment is
  not formatted with the lower part options.
  Also, it is not put into a minipage and it may span several pages.
  The distance between box and comment is controlled by \refKey{/tcb/before}.
\end{docTcbKey}


\clearpage
\subsection{Option Keys for Processing and Full Document Examples}\label{sec:proclistingkeys}
A complete \LaTeX\ document including |\documentclass|, |\begin{document}|
and |\end{document}| cannot be processed directly by |tcolorbox|.
It always has to be compiled separately. There are two methods supported
by the package to process and display such a full document example:
\begin{itemize}
\item Prepare and compile the example document independent from your main
  document. The source file and the resulting PDF file can be included
  into the main document afterwards. This is the most economic way since
  the example document can be left untouched after the example is complete.
\item The other possibility is to compile the example on the fly while
  the main document is compiled. This way has some charm, because the
  example can be edited inside the main document. But be aware that the
  compilation of the example is issued on every run of the main document.
  Also, there are fewer degrees of freedom how the example is compiled.
\end{itemize}

For both methods, the resulting example PDF file can be included
as a \refKey{/tcb/pdf comment}.

The following example shows how to apply the first method.
There already is a file |tcolorbox-example.tex| and a PDF file
|tcolorbox-example.pdf|. Both of them are input partly by the following:

\begin{dispListing}
% \tcbuselibrary{breakable,skins,raster}
\tcbinputlisting{
  enhanced jigsaw,breakable,pad at break*=2mm,height fixed for=first and middle,
  lower separated=false,
  leftlower=0pt,rightlower=0pt,middle=0pt,
  colframe=red!50!black,colback=yellow!10!white,
  listing and comment,
  listing file={tcolorbox-example},
  listing options=
    {style=tcblatex,texcsstyle=*\color{red!70!black},firstline=20,lastline=85},
  after upper={\par\bigskip\texttt{\ldots}\par},
  pdf comment,
  comment style={drop lifted shadow,graphics pages={1,...,4}},
}
\end{dispListing}
{\tcbusetemp}


\clearpage


\begin{docTcbKey}[][doc new=2014-11-14]{no process}{}{no default}
  Removes all processing commands if set before.
\end{docTcbKey}

\begin{docTcbKey}[][doc new=2014-11-14]{process code}{=\meta{code}}{no default, initially empty}
  Adds \meta{code} which is executed during \refCom{tcbinputlisting}
  and \refEnv{tcblisting}. At the time of executing the given \meta{code},
  the listing is already written to \refKey{/tcb/listing file}, but
  the colored box is not constructed yet.
  Its intended use is to process the listing somehow before displaying.
  The processing result can be used inside a \refKey{/tcb/comment}.
  Several \refKey{/tcb/process code} options can be given which are
  processed in the given order.
  Typically, \meta{code} is added by using the following styles
  \refKey{/tcb/run system command}, \refKey{/tcb/run pdflatex}, etc.
\end{docTcbKey}

\begin{marker}
To use the further options, the compiler has to be called with the
|-shell-escape| permission to authorize potentially dangerous system calls.
Be warned that this is a security risk. Anyway, it's more economic to
compile examples independent from the main document and to include them as
shown in the previous pages.
\end{marker}

\begin{docTcbKey}[][doc new=2014-11-14]{run system command}{=\meta{system command}}{style, no default, initially unset}
Runs a \meta{system command}, if the document is compiled with the
|-shell-escape| permission.
The current listing file can be accessed as |\filename@area\filename@base\filename@ext|.
This \meta{system command} is added to \refKey{/tcb/process code}.
\end{docTcbKey}


\begin{docTcbKey}[][doc new and updated={2014-11-14}{2016-05-27}]{compilable listing}{}{style, no default}
  Sets \refKey{/tcb/listing file} to |\jobname|\texttt{-listing-}\meta{counter}.
  \begin{marker}
  The default \refKey{/tcb/listing file} setting cannot be used to compile
  a listing, since the base name equals the |\jobname| and the included PDF
  files should be unique.
  Therefore, to use \refKey{/tcb/run pdflatex} etc., the \refKey{/tcb/listing file}
  has to be set to a unique value.
  One may use \refKey{/tcb/compilable listing} for this purpose.
  \end{marker}
\end{docTcbKey}


\begin{docTcbKey}[][doc new=2014-11-14]{run pdflatex}{\colOpt{=\meta{arguments}}}{style, no default, initially unset}
Issues a |pdflatex| compilation of the listing with the given \meta{arguments}.
\begin{itemize}
\item The main document has to be compiled with the |-shell-escape| permission.
\item The \refKey{/tcb/listing file} has to be unique for the listing.
\item If the listing has to be compiled twice, add |run pdflatex| two times to
  the option list.
\end{itemize}

\begin{dispListing*}{breakable,enhanced,pad at break*=2mm,before upper=,after skip=3mm}
\begin{tcblisting}{enhanced jigsaw,lower separated=false,
  leftlower=0pt,rightlower=0pt,
  colframe=red!50!black,colback=yellow!10!white,
  listing options={style=tcblatex,texcsstyle=*\color{red!70!black}},
  listing and comment,
  pdf comment,freeze pdf,
  compilable listing,
  run pdflatex
}
\documentclass{beamer}
\usetheme{Warsaw}
\begin{document}
\begin{frame}{Beamer example}
  \begin{block}{Hello World}
    \begin{itemize}[<+->]
      \item One
      \item Two
    \end{itemize}
  \end{block}

  \begin{alertblock}{Integral}
    \begin{equation}
      \visible<3->{\int\limits_1^x \frac{1}{t}~dt}
      \visible<4->{ = \ln(x).}
    \end{equation}
  \end{alertblock}
\end{frame}
\end{document}
\end{tcblisting}
\end{dispListing*}
{\tcbusetemp}
\end{docTcbKey}


\clearpage
\begin{docTcbKey}[][doc new=2014-11-14]{run xelatex}{\colOpt{=\meta{arguments}}}{style, no default, initially unset}
Issues a |xelatex| compilation of the listing with the given \meta{arguments}.
\end{docTcbKey}

\begin{docTcbKey}[][doc new=2014-11-14]{run lualatex}{\colOpt{=\meta{arguments}}}{style, no default, initially unset}
Issues a |lualatex| compilation of the listing with the given \meta{arguments}.
\end{docTcbKey}

\begin{docTcbKey}[][doc new=2014-11-14]{run makeindex}{\colOpt{=\meta{arguments}}}{style, no default, initially unset}
Issues a |makeindex| compilation of the listing with the given \meta{arguments}.
\end{docTcbKey}

\begin{docTcbKey}[][doc new=2014-11-14]{run bibtex}{\colOpt{=\meta{arguments}}}{style, no default, initially unset}
Issues a |bibtex| compilation of the listing with the given \meta{arguments}.
\end{docTcbKey}

\begin{docTcbKey}[][doc new=2014-11-14]{run biber}{\colOpt{=\meta{arguments}}}{style, no default, initially unset}
Issues a |biber| compilation of the listing with the given \meta{arguments}.
\end{docTcbKey}

\begin{docTcbKey}[][doc new=2014-11-14]{run arara}{\colOpt{=\meta{arguments}}}{style, no default, initially unset}
Issues an |arara| compilation of the listing with the given \meta{arguments}.
\end{docTcbKey}

\begin{docTcbKey}[][doc new=2014-11-14]{run latex}{\colOpt{=\meta{arguments}}}{style, no default, initially unset}
Issues a |latex| compilation of the listing with the given \meta{arguments}.
\end{docTcbKey}

\begin{docTcbKey}[][doc new=2014-11-14]{run dvips}{\colOpt{=\meta{arguments}}}{style, no default, initially unset}
Issues a |dvips| compilation of the listing with the given \meta{arguments}.
\end{docTcbKey}

\enlargethispage*{1cm}
\begin{docTcbKey}[][doc new=2014-11-14]{run ps2pdf}{\colOpt{=\meta{arguments}}}{style, no default, initially unset}
Issues a |ps2pdf| compilation of the listing with the given \meta{arguments}.
\end{docTcbKey}

\begin{dispListing*}{breakable,enhanced,pad at break*=2mm,before upper=,after skip=3mm}
\begin{tcblisting}{enhanced jigsaw,
  title={PSTricks with pdflatex},fonttitle=\bfseries,
  colframe=red!50!black,colback=yellow!10!white,
  listing options={style=tcblatex,texcsstyle=*\color{red!70!black}},
  lower separated=false,middle=0pt,
  listing side comment,righthand width=4cm,
  compilable listing,
  run latex,run dvips,run ps2pdf,
  pdf comment,freeze pdf,
  comment style={raster columns=1,
    graphics options={viewport=0.5in 7.7in 3.5in 10.5in,clip}},
}
\documentclass{article}
\usepackage{pstricks,multido}
\begin{document}
\psset{unit=3}%
\multido{\nHue=0.01+0.01}{100}{%
\definecolor{MyColor}{hsb}{\nHue,1,1}%
\pscircle[linewidth=0.01,linecolor=MyColor]{\nHue}}
\end{document}
\end{tcblisting}
\end{dispListing*}
{\tcbusetemp}

\clearpage

\begin{marker}
  For most applications, you will like to add \refKey{/tcb/freeze pdf} as option,
  since the included |pdf| file is only refreshed, if the source for this file
  has changed.
\end{marker}

\begin{docTcbKey}[][doc new=2016-07-14]{freeze file}{=\meta{file}}{no default, initially unset}
  Observes some \meta{file}, usually the final file produced by \refKey{/tcb/process code},
  \refKey{/tcb/run system command}, \refKey{/tcb/run pdflatex}, etc.
  If the MD5 checksum of the current \refKey{/tcb/listing file} is unchanged
  and \meta{file} exists, the processing is skipped and
  the \meta{file} is kept (frozen).
  Typically, the style \refKey{/tcb/freeze pdf} can be used for
  convenience.
\end{docTcbKey}

\begin{docTcbKey}[][doc new=2016-07-14]{freeze none}{}{no default, initially set}
  Freeze no file and always execute the given process commands.
\end{docTcbKey}

\begin{docTcbKey}[][doc new=2016-07-14]{freeze extension}{=\meta{text}}{style, no default}
  Calls \refKey{/tcb/freeze file} with the current \refKey{/tcb/listing file}
  stripped with its extension plus \meta{text} as new extension.
\begin{dispListing}
  ...
  listing file=myfile.tex,
  freeze extension=-modified.pdf,    % ->   myfile-modified.pdf   is observed
  ...
\end{dispListing}
\end{docTcbKey}

\begin{docTcbKey}[][doc new=2016-07-14]{freeze pdf}{}{no value}
  Calls \refKey{/tcb/freeze file} with the current \refKey{/tcb/listing file}
  stripped with its extension plus |.pdf| as new extension.
\end{docTcbKey}

\begin{docTcbKey}[][doc new=2016-07-14]{freeze png}{}{no value}
  Calls \refKey{/tcb/freeze file} with the current \refKey{/tcb/listing file}
  stripped with its extension plus |.png| as new extension.
  See the examples for \refKey{/tcb/run pdflatex} and \refKey{/tcb/run ps2pdf}.
\end{docTcbKey}

\begin{docTcbKey}[][doc new=2016-07-14]{freeze jpg}{}{no value}
  Calls \refKey{/tcb/freeze file} with the current \refKey{/tcb/listing file}
  stripped with its extension plus |.jpg| as new extension.
\end{docTcbKey}




\clearpage
\subsection{Creation of \LaTeX\ Tutorials}\label{sec:latextutorial}
The following source code gives a guideline for the creation of \LaTeX\ tutorials.
In the next section, a framework for \LaTeX\ exercises is described.
All examples shall be numbered optionally.

Firstly, some additional |tcb| keys are defined for the appearance.
For the examples, three environments |texexp|, |texexptitled|,
and |texexptitledspec| are defined with automatic numbering.

\begin{itemize}
\item |texexp| is used for untitled examles,
\item |texexptitled| is used for titled examles,
\item |texexptitledspec| is used for titled examles with special treatment.
\end{itemize}

\inputpreamblelisting{D}

\begin{dispExample}
\begin{tcblisting}{texexp}
This is a \LaTeX\ example which displays the text as source code
and in compiled form.
\end{tcblisting}
\end{dispExample}


\begin{dispExample}
\begin{texexptitled}{First example with a title line}{firstExample}
Here, we use Example \ref{firstExample} with a title line.
\end{texexptitled}
\end{dispExample}


\begin{dispExample}
\begin{texexp}{}
This is a \LaTeX\ example which displays the text as source code
and in compiled form.
\end{texexp}
\end{dispExample}


\begin{dispExample}
\begin{texexp}{text and listing}
This is a \LaTeX\ example which displays the text as source code
and in compiled form.
\end{texexp}
\end{dispExample}


\begin{dispExample}
\begin{texexp}{listing only}
This is a \LaTeX\ example which displays the text as source code only.
\end{texexp}
\end{dispExample}


\begin{dispExample}
\begin{texexp}{text only}
This is a \LaTeX\ example which displays the text in compiled form only.
\end{texexp}
\end{dispExample}


\begin{dispExample}
\begin{texexptitled}{An Example with a Heading}{heading1}
This is a \LaTeX\ example with a numbered heading line
which can be referred to.
\end{texexptitled}
Here, we see Example \ref{heading1}.
\end{dispExample}


\begin{dispExample}
\begin{texexptitled}[listing only]{Another Example with a Heading}{heading2}
The keys can be used in combination. Here, an example with a heading line
and source code only is given.
\end{texexptitled}
Here, we see Example \ref{heading2}.
\end{dispExample}


\begin{dispListing}
\begin{texexptitled}[float]{A floating Example with a Heading}{heading3}
This is another \LaTeX\ example with numbered heading line.
But now, the box is a floating object.
\end{texexptitled}
\end{dispListing}
\tcbusetemp

\begin{dispExample}
The floating box of the last example is seen as Example \ref{heading3}
on page \pageref{heading3}.
\end{dispExample}


\begin{dispExample}
\begin{texexptitledspec}{Special application}{texexpbox1}
\begin{lstlisting}[style=tcblatex]
Some \LaTeX\ source code.
\end{lstlisting}
\tcblower
For special cases, the environment |texexptitledspec| with style
|example| can be used directly. As one can see, the upper and the lower
part of the box can be used uncoupled also.
\end{texexptitledspec}
\end{dispExample}


The following series of examples demonstrate the application of
\refEnv{tcolorbox} options for diversification.

\begin{dispExample}
\begin{texexptitled}{How to use options (1):\par The basic example}{options1}
\begin{tikzpicture}
\path[fill=yellow!50!white] (0,0) circle (11mm);
\path[fill=white] (0,0) circle (9mm);
\foreach \w/\c in {90/red,210/green,330/blue}
{\path[shading=ball,ball color=\c] (\w:1cm) circle (7mm);}
\end{tikzpicture}
\end{texexptitled}
\end{dispExample}


\begin{dispExample}
\begin{texexptitled}[center lower,enhanced,segmentation hidden,middle=0mm]
  {How to use options (2):\par The text output is centered and the
    segmentation line has vanished.}{options2}
\begin{tikzpicture}
\path[fill=yellow!50!white] (0,0) circle (11mm);
\path[fill=white] (0,0) circle (9mm);
\foreach \w/\c in {90/red,210/green,330/blue}
{\path[shading=ball,ball color=\c] (\w:1cm) circle (7mm);}
\end{tikzpicture}
\end{texexptitled}
\end{dispExample}

\begin{dispExample}
\begin{texexptitled}[tikz lower,bicolor,colbacklower=white]
  {How to use options (3):\par Here, the |tikzpicture| is totally hidden.
    The |bicolor| skin highlights the output.}{options3}
\path[fill=yellow!50!white] (0,0) circle (11mm);
\path[fill=white] (0,0) circle (9mm);
\foreach \w/\c in {90/red,210/green,330/blue}
{\path[shading=ball,ball color=\c] (\w:1cm) circle (7mm);}
\end{texexptitled}
\end{dispExample}

\begin{dispExample}
\begin{texexptitled}[center lower,listing side text,righthand width=3.5cm,
  bicolor,colbacklower=white]
  {How to use options (4):\par The |bicolor| skin also works with side
   by side mode}{options4}
\begin{tikzpicture}
\path[fill=yellow!50!white] (0,0) circle (11mm);
\path[fill=white] (0,0) circle (9mm);
\foreach \w/\c in {90/red,210/green,330/blue}
{\path[shading=ball,ball color=\c]
    (\w:1cm) circle (7mm);}
\end{tikzpicture}
\end{texexptitled}
\end{dispExample}


\begin{dispExample}
\begin{texexptitled}[center lower,listing outside text,righthand width=3.5cm]
  {How to use options (5):\par Putting our picture outside is just
    a matter of one word.}{options5}
\begin{tikzpicture}
\path[fill=yellow!50!white] (0,0) circle (11mm);
\path[fill=white] (0,0) circle (9mm);
\foreach \w/\c in {90/red,210/green,330/blue}
{\path[shading=ball,ball color=\c]
    (\w:1cm) circle (7mm);}
\end{tikzpicture}
\end{texexptitled}
\end{dispExample}


\begin{dispExample}
\begin{texexptitled}[center lower,text above listing]
  {How to use options (6):\par The picture may also be put above
    the listing box.}{options6}
\begin{tikzpicture}
\path[fill=yellow!50!white] (0,0) circle (11mm);
\path[fill=white] (0,0) circle (9mm);
\foreach \w/\c in {90/red,210/green,330/blue}
{\path[shading=ball,ball color=\c]
    (\w:1cm) circle (7mm);}
\end{tikzpicture}
\end{texexptitled}
\end{dispExample}


\begin{dispExample}
\begin{texexptitled}[beamer,center lower,text outside listing,lefthand width=3.5cm]
  {How to use options (7):\par Our style is easily transformed into
    a beamerish one.}{options7}
\begin{tikzpicture}
\path[fill=yellow!50!white] (0,0) circle (11mm);
\path[fill=white] (0,0) circle (9mm);
\foreach \w/\c in {90/red,210/green,330/blue}
{\path[shading=ball,ball color=\c]
    (\w:1cm) circle (7mm);}
\end{tikzpicture}
\end{texexptitled}
\end{dispExample}


\clearpage
\subsection{Creation of \LaTeX\ Exercises}\label{listing:exercises}
In the following, a guideline is given for the creation of \LaTeX\ exercises
with solutions. These solutions are saved to disk for application at a place of
choice.
Therefore, all used exercises are logged to a file |\jobname.records| for automatic
processing. The solution contents themselves are saved to a subdirectory named
|solutions|. Also see \Vref{sec:recording}.

\begin{itemize}
\item Before the first exercise is given,
  \refCom{tcbstartrecording} has to be called to start recording.
\item The solution is given as content of a \refEnv{tcboutputlisting}
  environment. Note, that you can use this content also inside the
  exercise with \refCom{tcbuselistingtext} in compiled form.
\item After the last exercise is given (and before using the solutions),
  \refCom{tcbstoprecording} has to be called to stop recording.
\item The solutions are loaded by \refCom{tcbinputrecords}.
\end{itemize}

Inside the exercise text, there may be text parts which are needed as
\LaTeX\ source code and as compiled text as well. These parts can be
saved by \refEnv{tcbwritetemp} and used in compiled form by \refCom{tcbusetemp}
or as source code by \refCom{tcbusetemplisting}.

At first, we generate some a common style for the exercises and the
solutions. Further, since exercises and solutions should
be numbered, we force to use a label \meta{marker}.
Automatically, the label |exe:|\meta{marker} is used to mark the exercise
and the label |sol:|\meta{marker} is used to mark the solution.

\begin{dispListing}
\tcbset{texercisestyle/.style={arc=0.5mm, colframe=blue!25!yellow!90!white,
  colback=blue!25!yellow!5!white, coltitle=blue!25!yellow!40!black,
  fonttitle=\small\sffamily\bfseries, fontupper=\small, fontlower=\small,
  listing options={style=tcblatex,texcsstyle=*\color{red!40!black}},
}}
\end{dispListing}
\tcbusetemp

With these preparations, the kernel environment |texercise| for our
exercises is created quickly:

\inputpreamblelisting{E}

\clearpage
The following examples demonstrate the application.

\begin{dispListing}
\tcbstartrecording
\end{dispListing}
\tcbusetemp


\begin{dispExample}
\begin{texercise}{tabular_example}
\textit{Create the following table:}\par\smallskip%
\begin{tcboutputlisting}
\begin{tabular}{|p{3cm}|p{3cm}|p{3cm}|p{3cm}|}\hline
\multicolumn{4}{|c|}{\bfseries\itshape Das alte Italien}\\\hline
\multicolumn{2}{|c|}{\bfseries Antike} &
\multicolumn{2}{c|}{\bfseries Mittelalter}\\\hline
\multicolumn{1}{|c|}{\itshape Republik}&
\multicolumn{1}{c|}{\itshape Kaiserreich}&
\multicolumn{1}{c|}{\itshape Franken}&
\multicolumn{1}{c|}{\itshape Teilstaaten}\\\hline
In den Zeiten der r\"{o}mischen Republik standen dem Staat jeweils zwei
Konsuln vor, deren Machtbefugnisse identisch waren. &
Das r\"{o}mische Kaiserreich wurde von einem Alleinherrscher, dem Kaiser,
regiert.
& In der V\"{o}lkerwanderungszeit \"{u}bernahmen die Goten und sp\"{a}ter die
Franken die Vorherrschaft.
& Im sp\"{a}teren Mittelalter regierten F\"{u}rsten einen Fleckenteppich
von Einzelstaaten.\\\hline
\end{tabular}
\end{tcboutputlisting}
\tcbuselistingtext%
\end{texercise}
\end{dispExample}


\begin{dispExample}
\begin{texercise}{macro_oneparam}
\begin{tcboutputlisting}
\newcommand{\headingline}[1]{%
  \begin{center}\Large\bfseries #1\end{center}}
\end{tcboutputlisting}
\tcbuselistingtext%

Create a new macro \verb+\headingline+ which produces the
following output:\par\smallskip
\begin{tcbwritetemp}
\headingline{Very important heading}
\end{tcbwritetemp}
\tcbusetemplisting\tcbusetemp%
\end{texercise}
\end{dispExample}



\begin{dispExample}
\begin{texercise}{macro_twoparam}
\begin{tcboutputlisting}
\newcommand{\minitable}[2]{%
  \begin{center}\begin{tabular}{p{10cm}}\hline%
  \multicolumn{1}{c}{\bfseries#1}\\\hline%
  #2\\\hline%
  \end{tabular}\end{center}}
\end{tcboutputlisting}
\tcbuselistingtext%
Create a new macro \verb+\minitable+ which produces the
following output:\par\smallskip
\begin{tcbwritetemp}
\minitable{My heading}{In this tiny tabular, there is only a heading
  and some text below which has a width of ten centimeters.}
\end{tcbwritetemp}
\tcbusetemplisting\par\smallskip\tcbusetemp%
\end{texercise}
\end{dispExample}


\begin{dispExample}
\begin{texercise}{macro_threeparam}
\begin{tcboutputlisting}
\newcommand{\synop}[3]{%
  \begin{tabular}{@{}p{(\linewidth-\tabcolsep*2-\arrayrulewidth)/2}|%
    p{(\linewidth-\tabcolsep*2-\arrayrulewidth)/2}@{}}\hline
  \multicolumn{2}{c}{\bfseries #1}\\\hline
  \multicolumn{1}{c|}{\itshape English}&
  \multicolumn{1}{c}{\itshape German}\\\hline
  #2 & #3
  \end{tabular}}
\end{tcboutputlisting}
\tcbuselistingtext%
Create a new macro \verb+\synop+ which typesets a synoptic text according
to the following example. Base your macro on a tabular which takes the
total line width.\par\smallskip
\begin{tcbwritetemp}
\synop{Neil Armstrong}%
{That's one small step for a man, one giant leap for mankind.}%
{Das ist ein kleiner Schritt f\"{u}r einen Mann,
   ein riesiger Sprung f\"{u}r die Menschheit.}
\end{tcbwritetemp}
\tcbusetemplisting\par\smallskip\tcbusetemp%
\end{texercise}
\end{dispExample}
%\closeoutsol

\begin{dispListing}
\tcbstoprecording
\end{dispListing}
\tcbusetemp

\bigskip

Now, we give a list of all exercises with:

\begin{dispListing}
\tcblistof[\subsection]{exam}{List of Exercises%
  \label{listofexercises}}
\end{dispListing}
\tcbusetemp

\clearpage
\subsection{Solutions for the given \LaTeX\ Exercises}

For all solutions, a macro |\processsol| was written to the file |\jobname.records|.
Now, we need a definition for this macro to use the solutions.

\begin{dispListing}
% \usepackage{hyperref} % for phantomlabel
\newtcbinputlisting{\processsol}[2]{%
  texercisestyle,
  listing only,
  listing file={#1},
  phantomlabel={sol:#2},%
  title={Solution for Exercise \ref{exe:#2} on page \pageref{exe:#2}},
}
\end{dispListing}
\tcbusetemp

The loading of all solutions is done by:

\begin{dispListing}
\tcbinputrecords
\end{dispListing}

With this, we get:

\tcbusetemp



